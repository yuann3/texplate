\documentclass[12pt]{article}
\usepackage[utf8]{inputenc}
\usepackage{geometry}
\geometry{left=3cm,right=3cm,top=2cm,bottom=2cm}
\usepackage{hyperref}
\usepackage[backend=biber, style=authoryear, citestyle=authoryear]{biblatex}  % Ensure BibLaTeX is configured for biber backend

% Define \tightlist for Pandoc-generated lists
\providecommand{\tightlist}{%
  \setlength{\itemsep}{0pt}\setlength{\parskip}{0pt}}

% Add the bibliography resource (adjust path as necessary)
% Ensure this path points to where your BibTeX file is relative to where you compile the LaTeX file
\addbibresource{bib/references.bib}

\title{\textbf{\LARGE \textbf{MongoDB: A Comprehensive Analysis of its NoSQL Architecture, Features, and Applications in Big Data Environments}}}
\author{Yiyuan Li \\ C3434681}
\date{\today}

\begin{document}

\maketitle

\begin{center}
  Advanced Database \\
  School of Information and Physical Sciences \\
  University of Newcastle, Australia \\
\end{center}

\begin{abstract}
This report focuses on MongoDB, a contemporary document-based NoSQL
database system specifically designed to handle vast quantities of
unstructured data. The primary objective of this analysis is to provide
a comprehensive overview of MongoDB's features, architectural design,
and functional capabilities. Additionally, it aims to evaluate the
system's applicability in scenarios involving the management of large
datasets, while addressing the challenges associated with such
processes. The document presents a detailed examination of
document-based data modeling and storage architecture, horizontal
scalability, advanced querying capabilities, eventual consistency model,
and mechanisms of fault tolerance. This analysis includes a literature
review and a practical proof-of-concept demonstration for CRUD
operations to understand the strengths and use cases of MongoDB.
\end{abstract}

\newpage

\section{Introduction}\label{introduction}

In the era of big data, traditional RDBMS are naturally very limited in
handling this kind of modern data. The challenge spawned a number of
NoSQL databases designed to meet scalability and flexibility
requirements; among them is MongoDB with its document-oriented way of
storing and retrieving data.

MongoDB is developed in 2009 by MongoDB Inc., stores data in flexible,
JSON-like documents. It provides an effective way of handling
unstructured and semi-structured data, quite different from the rigid
table structure of traditional relational databases.

This report is an overview of MongoDB, by giving insight into its
architecture, features, and applications within big data environments.
Precisely, this report has the following objectives:

\begin{itemize}
\tightlist
\item
  Understand MongoDB's document-based data model and its differentiating
  features.
\item
  Understand the reasons for the development of MongoDB, its benefits
  and shortcomings against relational databases.
\item
  Analyze the MongoDB Storage Architecture: their approach to scaling.
\item
  Know what kind of query capabilities MongoDB offers and how they
  differ from traditional SQL.
\item
  Critically assess the concurrency control approach of MongoDB, and how
  it balances consistency with availability and partition tolerance
  using the CAP theorem.
\item
  Examine fault tolerance mechanisms of MongoDB and their forbearance on
  data reliability. Show basic CRUD (Create, Read, Update, Delete)
  operations that can be done in MongoDB to demonstrate its practical
  application.
\end{itemize}

\section{Data Model and Features}\label{data-model-and-features}

\section{References}\label{references}

{[}0{]}I did your mom - yiyuan li - 2024 % Main content

\printbibliography[heading=bibintoc, title={References}] % Print the bibliography

\end{document}
